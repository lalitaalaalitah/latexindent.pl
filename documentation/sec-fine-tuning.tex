% arara: pdflatex: {shell: yes, files: [latexindent]}
\section{Fine tuning}\label{sec:finetuning}
 \texttt{latexindent.pl} operates by looking for the code blocks detailed in
 \vref{tab:code-blocks}.
 \announce{2019-07-13}{details of fine tuning of code blocks} The fine tuning of the details of such code blocks
 is controlled by the \texttt{fineTuning} field, detailed in \cref{lst:fineTuning}.

 This field is for those that would like to peek under the bonnet/hood and make some fine
 tuning to \texttt{latexindent.pl}'s operating.
 \index{warning!fine tuning}
 \index{regular expressions!fine tuning}
 \index{regular expressions!environments}
 \index{regular expressions!ifElseFi}
 \index{regular expressions!commands}
 \index{regular expressions!keyEqualsValuesBracesBrackets}
 \index{regular expressions!NamedGroupingBracesBrackets}
 \index{regular expressions!UnNamedGroupingBracesBrackets}
 \index{regular expressions!arguments}
 \index{regular expressions!modifyLineBreaks}
 \index{regular expressions!lowercase alph a-z}
 \index{regular expressions!uppercase alph A-Z}
 \index{regular expressions!numeric 0-9}
 \index{regular expressions!at least one +}
 \index{regular expressions!horizontal space \textbackslash{h}}

 \begin{warning}
	 Making changes to the fine tuning may have significant consequences for your indentation scheme,
	 proceed with caution!
 \end{warning}

 \begin{widepage}
	 \cmhlistingsfromfile[style=fineTuning]*{../defaultSettings.yaml}[width=0.95\linewidth,before=\centering,yaml-TCB]{\texttt{fineTuning}}{lst:fineTuning}
 \end{widepage}

 The fields given in \cref{lst:fineTuning} are all \emph{regular expressions}. This manual
 is not intended to be a tutorial on regular expressions; you might like to read, for
 example, \cite{masteringregexp} for a detailed covering of the topic.

 We make the following comments with reference to \cref{lst:fineTuning}:
 \begin{enumerate}
	 \item the \texttt{environments:name} field details that the \emph{name} of an
	       environment can contain:
	       \begin{enumerate}
		       \item \texttt{a-z} lower case letters
		       \item \texttt{A-Z} upper case letters
		       \item \texttt{@} the \texttt{@} 'letter'
		       \item \lstinline!\*! stars
		       \item \texttt{0-9} numbers
		       \item \lstinline!_! underscores
		       \item \lstinline!\! backslashes
	       \end{enumerate}
	       \index{regular expressions!at least one +}
	       The \texttt{+} at the end means \emph{at least one} of the above
	       characters.
	 \item the \texttt{ifElseFi:name} field:
	       \begin{enumerate}
		       \item \lstinline^@?^ means that it \emph{can possibly} begin with
		             \lstinline^@^
		       \item followed by \texttt{if}
		       \item followed by 0 or more characters from \texttt{a-z}, \texttt{A-Z} and
		             \texttt{@}
		       \item the \texttt{?} the end means \emph{non-greedy}, which means `stop the
		             match as soon as possible'
	       \end{enumerate}
	 \item the \texttt{keyEqualsValuesBracesBrackets} contains some interesting syntax:
	       \begin{enumerate}
		       \item \lstinline!|! means `or'
		       \item \lstinline^(?:(?<!\\)\{)^ the \lstinline^(?:...)^ uses a \emph{non-capturing} group --
		             you don't necessarily need to worry about what this means, but just know that for the
		             \texttt{fineTuning} feature you should only ever use \emph{non}-capturing
		             groups, and \emph{not} capturing groups, which are simply
		             \lstinline!(...)!
		       \item \lstinline^(?<!\\)\{)^ means a \lstinline^{^ but it can \emph{not}
		             be immediately preceded by a \lstinline!\!
	       \end{enumerate}
	 \item in the \texttt{arguments:before} field
	       \begin{enumerate}
		       \item \lstinline^\d\h*^ means a digit (i.e. a number), followed by 0 or more horizontal
		             spaces
		       \item \lstinline^;?,?^ means \emph{possibly} a semi-colon, and possibly a comma
		       \item \lstinline^\<.*?\>^ is designed for 'beamer'-type commands; the
		             \lstinline^.*?^ means anything in between \lstinline^<...>^
	       \end{enumerate}
	 \item the \texttt{modifyLineBreaks} field refers to fine tuning settings detailed in
	       \vref{sec:modifylinebreaks}. In particular:
	       \begin{enumerate}
		       \item \texttt{betterFullStop} is in relation to the one sentence per line routine, detailed in
		             \vref{sec:onesentenceperline}
		       \item \texttt{doubleBackSlash} is in relation to the \texttt{DBSStartsOnOwnLine} and
		             \texttt{DBSFinishesWithLineBreak} polyswitches surrounding double back slashes, see
		             \vref{subsec:dbs}
		       \item \texttt{comma} is in relation to the \texttt{CommaStartsOnOwnLine} and
		             \texttt{CommaFinishesWithLineBreak} polyswitches surrounding commas in optional and mandatory
		             arguments; see \vref{tab:poly-switch-mapping}
	       \end{enumerate}
 \end{enumerate}

 It is not obvious from \cref{lst:fineTuning}, but each of the \texttt{follow},
 \texttt{before} and \texttt{between} fields allow trailing comments, line
 breaks, and horizontal spaces between each character.

 \begin{example}
	 As a demonstration, consider the file given in \cref{lst:finetuning1}, together with its
	 default output using the command
	 \begin{commandshell}
latexindent.pl finetuning1.tex 
\end{commandshell}
	 is given in \cref{lst:finetuning1-default}.

	 \begin{cmhtcbraster}[raster column skip=.01\linewidth]
		 \cmhlistingsfromfile{demonstrations/finetuning1.tex}{\texttt{finetuning1.tex}}{lst:finetuning1}
		 \cmhlistingsfromfile{demonstrations/finetuning1-default.tex}{\texttt{finetuning1.tex} default}{lst:finetuning1-default}
	 \end{cmhtcbraster}

	 It's clear from \cref{lst:finetuning1-default} that the indentation scheme has not worked as
	 expected. We can \emph{fine tune} the indentation scheme by employing the settings
	 given in \cref{lst:fine-tuning1} and running the command
	 \index{switches!-l demonstration}
	 \begin{commandshell}
latexindent.pl finetuning1.tex -l=fine-tuning1.yaml
\end{commandshell}
	 and the associated (desired) output is given in \cref{lst:finetuning1-mod1}.
	 \index{regular expressions!at least one +}

	 \begin{cmhtcbraster}[raster column skip=.01\linewidth]
		 \cmhlistingsfromfile{demonstrations/finetuning1-mod1.tex}{\texttt{finetuning1.tex} using \cref{lst:fine-tuning1}}{lst:finetuning1-mod1}
		 \cmhlistingsfromfile[style=yaml-LST]*{demonstrations/fine-tuning1.yaml}[yaml-TCB]{\texttt{finetuning1.yaml}}{lst:fine-tuning1}
	 \end{cmhtcbraster}
 \end{example}

 \begin{example}
	 Let's have another demonstration; consider the file given in \cref{lst:finetuning2}, together with its
	 default output using the command
	 \begin{commandshell}
latexindent.pl finetuning2.tex 
\end{commandshell}
	 is given in \cref{lst:finetuning2-default}.

	 \begin{cmhtcbraster}[raster column skip=.01\linewidth,
			 raster left skip=-3.75cm,
			 raster right skip=0cm,]
		 \cmhlistingsfromfile{demonstrations/finetuning2.tex}{\texttt{finetuning2.tex}}{lst:finetuning2}
		 \cmhlistingsfromfile{demonstrations/finetuning2-default.tex}{\texttt{finetuning2.tex} default}{lst:finetuning2-default}
	 \end{cmhtcbraster}

	 It's clear from \cref{lst:finetuning2-default} that the indentation scheme has not worked as
	 expected. We can \emph{fine tune} the indentation scheme by employing the settings
	 given in \cref{lst:fine-tuning2} and running the command
	 \index{switches!-l demonstration}
	 \begin{commandshell}
latexindent.pl finetuning2.tex -l=fine-tuning2.yaml
\end{commandshell}
	 and the associated (desired) output is given in \cref{lst:finetuning2-mod1}.

	 \begin{cmhtcbraster}[raster column skip=.01\linewidth,
			 raster left skip=-3.75cm,
			 raster right skip=0cm,]
		 \cmhlistingsfromfile{demonstrations/finetuning2-mod1.tex}{\texttt{finetuning2.tex} using \cref{lst:fine-tuning2}}{lst:finetuning2-mod1}
		 \cmhlistingsfromfile[style=yaml-LST]*{demonstrations/fine-tuning2.yaml}[yaml-TCB]{\texttt{finetuning2.yaml}}{lst:fine-tuning2}
	 \end{cmhtcbraster}

	 In particular, note that the settings in \cref{lst:fine-tuning2} specify that \texttt{NamedGroupingBracesBrackets}
	 and \texttt{UnNamedGroupingBracesBrackets} can follow \texttt{"} and that we allow \lstinline!---! between arguments.
 \end{example}

 \begin{example}
	 You can tweak the \texttt{fineTuning} using the \texttt{-y} switch, but to be sure to use quotes appropriately.
	 For example, starting with the code in \cref{lst:finetuning3} and running the following command
	 \begin{commandshell}
latexindent.pl -m -y='modifyLineBreaks:oneSentencePerLine:manipulateSentences: 1, modifyLineBreaks:oneSentencePerLine:sentencesBeginWith:a-z: 1, fineTuning:modifyLineBreaks:betterFullStop: "(?:\.|;|:(?![a-z]))|(?:(?<!(?:(?:e\.g)|(?:i\.e)|(?:etc))))\.(?!(?:[a-z]|[A-Z]|\-|~|\,|[0-9]))"' issue-243.tex -o=+-mod1
\end{commandshell}
	 gives the output shown in \cref{lst:finetuning3-mod1}.

	 \cmhlistingsfromfile*{demonstrations/finetuning3.tex}{\texttt{finetuning3.tex}}{lst:finetuning3}
	 \cmhlistingsfromfile*{demonstrations/finetuning3-mod1.tex}{\texttt{finetuning3.tex} using -y switch}{lst:finetuning3-mod1}
 \end{example}
